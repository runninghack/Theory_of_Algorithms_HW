\documentclass{article}
\usepackage[utf8]{inputenc}
\usepackage{amsmath}
\usepackage{algorithmicx}
\usepackage{algorithm}
\usepackage{algpseudocode}
\usepackage{indentfirst}

\algnewcommand\algorithmicinput{\textbf{//Input:}}
\algnewcommand\INPUT{\item[\algorithmicinput]}
\algnewcommand\algorithmicoutput{\textbf{//Output:}}
\algnewcommand\OUTPUT{\item[\algorithmicoutput]}



\title{Homework 2}
\author{Lei Zhang}

\begin{document}

\maketitle

\section{Exercises 2.3 - 1}

\subsection{a.}

\begin{align*}
1+ 3 + 5 + 7 + . . . + 999 &= \frac{(1+999)500}{2}\\
&= 250000
\end{align*}

\subsection{b.}

\begin{align*}
2 + 4 + 8 + 16 + . . . + 1024 &= \frac{2^{10+1}-1}{2-1} -1\\
&= 2047 - 1\\
&= 2046
\end{align*}

\subsection{c.}


\begin{align*}
\sum_{i=3}^{n+1}1 &= (n + 1) - 3 + 1 \\
&=  n-1
\end{align*}


\subsection{d.}

\begin{align*}
\sum_{i=3}^{n+1}i &=  3 + 4 + ... n+1\\
&=  (\sum_{i=1}^{n+1}i) - 1 -2\\
&=  \frac{(n+1)(n+2)}{2} -3\\
&=  \frac{n^2+3n-4}{2}
\end{align*}

\subsection{e.}

\begin{align*}
\sum_{i=0}^{n-1}i(i+1) &=  \sum_{i=0}^{n-1}i^2 + \sum_{i=0}^{n-1}i\\
&= \frac{(n-1)n(2n-1)}{6} + \frac{(n-1)n}{2} \\
&= \frac{(n^2-1)n}{3}
\end{align*}

\section{Exercises 2.3 - 2}

\subsection{a.}

\begin{align*}
\sum_{0}^{n-1}(i^2+1)^2&= \sum_{0}^{n-1}i^4 + 2\sum_{0}^{n-1}i^2 + \sum_{0}^{n-1}1 \\
&= \frac{1}{5}(n-1)^5+\frac{1}{3}(n-1)^3+ n-1 \\
&\approx \frac{1}{5}n^5  \\
&\in \Theta(n^5) \\
\end{align*}

\subsection{b.}

\begin{align*}
\sum_{i=2}^{n-1}\ln i^2&= 2\sum_{i=1}^{n-1}\ln i - 2\ln 1\\
&= 2(n-1)\ln(n-1)\\
&\approx 2n\ln n  \\
& \in \Theta (n \ln n)
\end{align*}

\subsection{c.}

\begin{align*}
\sum_{i=1}^{n}(i+1)2^{i-1}&= \frac{1}{4}\sum_{i=1}^{n}(i+1)2^{i+1}\\
&= \frac{1}{4}\sum_{k=0}^{n+1}k2^k\\
&= \frac{1}{4}\sum_{k=1}^{n+1}k2^k\\
&= \frac{n}{4}2^{n+2} +2 \\
& \approx \frac{n}{16}2^n \\
& \in\Theta (n2^n)
\end{align*}

\subsection{d.}

\begin{align*}
\sum_{i=0}^{n-1}\sum_{j=0}^{i-1}(i+j)&= \sum_{i=0}^{n-1}[i(i-1) + \sum_{j=0}^{i-1}j] \\
&= \sum_{i=0}^{n-1}[i^2 -i + \frac{(i-1)i}{2}]\\
&= \frac{3}{2}\sum_{i=0}^{n-1}(i^2 -i)\\
&= \frac{3}{2}\sum_{i=0}^{n-1}i^2 - \frac{3}{2}\sum_{i=0}^{n-1}i\\
&= \frac{3}{2}\frac{(n-1)n(2n-1)}{6} - \frac{3}{2}\frac{n(n-1)}{2}\\
&\approx \frac{n^3}{2}\\
&\in \Theta(n^3)
\end{align*}

\section{Exercises 2.3 - 4}

a.$\sum_{i=1}^{n}i^2$

b.$S \leftarrow S+ i*i$

c.$C(n) = \sum_{i=1}^n1=n$

d.$C(n) \in \Theta(n)$

e.It is known that $\sum_{i=1}^{n}i^2 = \frac{n(n+1)(2n+1)}{6}$, so the algorithm can use this formula to compute the results directly.


\section{Exercises 2.3 - 6}

a. If the input is a symmetric matrix, the algorithm returns true. If the matrix is not symmetric, the algorthm returns false.

b. The comparisons: $if A[i, j ] \ne A[j, i]$

c.

\begin{align*}
C(n) &= \sum_{i=0}^{n-2}\sum_{j=i+1}^{n-1}1\\
&= \sum_{i=0}^{n-2}[(n-1)-(i+1)+1]\\
&= \sum_{i=1}^{n-2}(n-1-i)\\
&= (n-1)(n-1) - \sum_{i=0}^{n-2}i\\
&= (n-1)^2 - \frac{(n-2)(n-1)}{2}\\
&= \frac{n(n-1)}{2}
\end{align*}

d. $C(n)\in \Theta(n^2)$

e. There are no better algorithms because any algorithms have to compare $\frac{n(n-1)}{2}$ times for this problem.

\section{Exercises 2.3 - 9}

\textbf{Basis}: For n = 1, the left-hand side of the equation is 1, and the right-hand side of the equation is $\frac{1(1+1)}{2} = 1$. Thus the statement is true for n = 1.

\textbf{Inductive step}: Assume $\sum_{i=1}^{k}i = \frac{k(k+1)}{2}$ holds for some unspecified value of k. Then  $\sum_{i=1}^{n+1}i = (n+1) + \sum_{i=1}^{n}i = (n+1) + \frac{n(n+1)}{2} = \frac{(n+1)(n+2)}{2}$

 Since both the basis and the inductive step have been performed, by mathematical induction, the statement $\sum_{i=1}^{n}i = \frac{n(n+1)}{2}$ holds for all natural numbers n. Q.E.D.




\end{document}
