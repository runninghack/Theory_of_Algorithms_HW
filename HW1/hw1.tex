\documentclass{article}
\usepackage[utf8]{inputenc}
\usepackage{amssymb}
\usepackage{amsmath}
\usepackage{amsthm}
\usepackage{algorithm}

\usepackage{algorithmic}
\usepackage{xcolor}
%\floatname{algorithm}{Procedure}
\renewcommand{\algorithmicrequire}{\textbf{Input:}}
\renewcommand{\algorithmicensure}{\textbf{Output:}}


\title{Homework 1}
\author{Lei Zhang}

\begin{document}

\maketitle

\section{Exercises 1.1 - 5}

\begin{algorithm}
\caption{Finding common elements in two sorted lists}
\begin{algorithmic}[1]
\REQUIRE ~~\
An array $A_1[0..m-1]$ for m sorted numbers;\\
Another sorted array $A_2[0..n-1]$ with n numbers;\\
\ENSURE ~~\
Ensemble of classifiers on the current batch, $A_c$; \\
\STATE $ i \Leftarrow 0 $
\WHILE {$ (i < m) and (j < n) $}
\IF {$A_1[i] == A_2[i]$}
\STATE $print A[i]$
\STATE $i \leftarrow i + 1$
\STATE $j \leftarrow j + 1$
\ELSIF{$A_1[i] > A_2[i] $}
\STATE $j \leftarrow j + 1$
\ELSE
\STATE $i \leftarrow i +1$
\ENDIF
\ENDWHILE
% \RETURN $l_c$;
\end{algorithmic}
\end{algorithm}

\section{Exercises 1.1 - 6}

\textbf{a.}

m = 31415, n = 14142

Step 1: $r = 31415 \mod 14142 = 3131, m \leftarrow 14142, n \leftarrow 3131 $

Step 2: $r = 14142 \mod 3131 = 1618, m \leftarrow 3131, n \leftarrow 1618 $

Step 3: $r = 3131 \mod 1618 = 1513, m \leftarrow 1618, n \leftarrow 1513 $

Step 4: $r = 1618 \mod 1513 = 105, m \leftarrow 1513, n \leftarrow 105 $

Step 5: $r = 1513 \mod 10 5= 43, m \leftarrow 105, n \leftarrow 43 $

Step 6: $r = 105 \mod 43 = 19, m \leftarrow 43, n \leftarrow 19 $

Step 7: $r = 43 \mod 19 = 5, m \leftarrow 19, n \leftarrow 5 $

Step 8: $r = 19 \mod 5 = 4, m \leftarrow 5, n \leftarrow 4 $

Step 9: $r = 5 \mod 4 = 1, m \leftarrow 4, n \leftarrow 1 $

Step 10: $r = 4 \mod 1 = 0, m \leftarrow 1, n \leftarrow 0 $

Step 11: return 1

\textbf{b.}

Using consecutive integer checking algorithm, every time the value of t is decreased, it takes two divisions. There are 28284 divisions in total.

Using Euclid's algorithm, there are 10 divisions.

Euclid's algorithm is 28284/10 = 2828.4 times faster in terms of divisions.

\section{Exercises 1.1 - 7}

Prove the equality gcd(m, n) = gcd(n, m mod n) for every pair of positive integers m and n.

\section{Exercises 1.2 - 4}

\begin{algorithm}
\caption{finding real roots}
\begin{algorithmic}[1]
\REQUIRE ~~\
arbitrary real coefficients a, b, and c
\STATE $ d \leftarrow b^2-4*a*c  $
\IF {$d<0$}
\RETURN $false$
\ELSIF{$d = 0$}
\STATE $ x_1 \leftarrow (-b + sqrt(d))/(2*a)$
\RETURN $x_1$
\ELSIF{$d>0$}
\STATE $x_1 \leftarrow (-b + sqrt(d))/(2*a)$
\STATE $x_2 \leftarrow (-b - sqrt(d))/(2*a)$
\RETURN $ x_1, x_2$
\ENDIF
\end{algorithmic}
\end{algorithm}

\section{Exercises 1.2 - 9}


\section{Exercises 1.3 - 1}



\end{document}
