\documentclass{article}
\usepackage[utf8]{inputenc}
\usepackage{amsmath}
\usepackage{algorithmicx}
\usepackage{algorithm}
\usepackage{algpseudocode}
\usepackage{indentfirst}

\algnewcommand\algorithmicinput{\textbf{//Input:}}
\algnewcommand\INPUT{\item[\algorithmicinput]}
\algnewcommand\algorithmicoutput{\textbf{//Output:}}
\algnewcommand\OUTPUT{\item[\algorithmicoutput]}

\title{Homework 5}
\author{Lei Zhang}

\begin{document}

\maketitle

\section{Exercises 5.1 - 1}

\subsection*{a}

\begin{center}
\begin{algorithmic}
\INPUT
An array A[0..n-1] of comparable elements
\OUTPUT
The number of the largest element in the array
\Function{Max}{$A[0..n-1]$}
\If {$n == 1$}
\State \Return {$0$}
\ElsIf {$A[Max(A[0..\lfloor n/2-1 \rfloor])] > A[Max(A[\lfloor n/2 \rfloor..n])]$}
\State \Return $Max(A[0..\lfloor n/2-1 \rfloor)])$
\Else
\State \Return $Max(A[\lfloor n/2 \rfloor..n])$
\EndIf
\EndFunction
\end{algorithmic}
\end{center}

\subsection*{b}

The output will be the leftmost largest element.

\subsection*{c}

$C(n) = 2C(\lfloor n/2 \rfloor) + 1$ for $ n > 1$, $C(1) = 0$

It is easy to find the exact solution to the worst-case recurrence for $n = 2^k$:

$C_{worst}(n) = nlog_2n-n+1$

\subsection*{d}

The divide-and-conquer algorithm and rute-force algorithm have the same time complexity. They also have exactly the same number of elementary operations.


\section{Exercises 5.1 - 2}

\subsection*{a}

\begin{center}
\begin{algorithmic}
\INPUT
An array A[0..n-1] of comparable elements
\OUTPUT
The smallest and largest elements in the array, $[min, max]$
\Function{MinAndMax}{$A[0..n-1]$}
\If {$n == 1$}
\State \Return {$[A[0], A[0]]$}
\ElsIf {$n == 2 \And$  $A[0] <= A[1]$}
\State \Return {$[A[0], A[1]]$}
\ElsIf {$n == 2 \And$  $A[0] > A[1]$}
\State \Return {$[A[1], A[0]]$}
\Else
\State $min,max = MinAndMax(A[0..\lfloor n/2-1 \rfloor)])$
\State $min2,max2 = MinAndMax(A[\lfloor n/2 \rfloor)..n])$
\If {$min2 < min$}
\State $min \leftarrow min2$
\EndIf
\If {$max2 < max$}
\State $max \leftarrow max2$
\EndIf
\State \Return {$[min, max]$}
\EndIf
\EndFunction
\end{algorithmic}
\end{center}


\subsection*{b}

$C(n) = 2C(n/2) + 2$ for $n > 2, C(2) = 1, C(1) = 0$

It is easy to find the exact solution to the worst-case recurrence for $n = 2^k$:

$C(n) = \frac{3}{2}n-2$

\subsection*{c}

The divide-and-conquer algorithm and brute-force algorithm have the same time complexity, but the divide-and-conquer algorithm algorithm make less (about 25\%) elementary operations than the brute-force algorithm.

\section{Exercises 5.1 - 5}

\subsection*{a}

$a = 4, b = 2, d = 1. a > b^d$

Thus, $ T(n) \in \Theta(n^{log_2^4}) = \Theta(n^2)$

\subsection*{b}

$a = 4, b=2, d = 2. a = b^d$

Thus, $T(n) \in \Theta(n^2logn)$

\subsection*{c}

$a = 4, b = 2, d = 3. a < b^d$

Thus, $T(n) \in \Theta(n^3)$

\section{Exercises 5.3 - 2}

It's not correct. The correct version:

\begin{center}
\begin{algorithmic}
\INPUT
A binary tree $T$
\OUTPUT
The number of leaves in $T$
\Function{LeafCounter}{$T$}
\If $T = \emptyset$
\State \Return $0$
\ElsIf {$ T_L = \emptyset \And T_R= \emptyset$}
\State \Return $1$
\Else
\State \Return $LeafCounter(T_L) + LeafCounter(T_R)$
\EndIf
\EndFunction
\end{algorithmic}
\end{center}


\section{Exercises 5.3 - 5}

\subsection*{a}

a, b, d, e, c, f

\subsection*{b}

d, b, e, a, c, f

\subsection*{c}

d, e, b, f, c, a

\section{Exercises 5.4 - 2}

\textbf{First Step:}

$a1 = 21, a0= 01, b1 = 11, b0= 30$

$c2 =  a1 * b1 = 21*11, c0 = a0*b0=01*30, c1 = (21 + 01) * (11+30)-(c2+c0) = 22*41-21*11-01*01*30$

\textbf{Second Step:}

For 21 * 11: $c1 = (2+1)*(1+1)-(2+1)=3$, thus, $21*11=2*10^2+3*10+1 = 231$

For 01 * 30: $c1 = (0+1) *(3+0)-(0+0) = 3$, thus, $01*30=0+3*10+0=30$

For 22*41: $c1=(2+2)*(4+1)-(8+2) = 10$, thus, $22*41 = 8*10^2+10*10+2 = 902$

\textbf{At Last:}

The result is: $2101*1130=231*10^4+(902-231-30)*10^2+30=2,374,130$

\section{Exercises 5.4 - 5}

$C(n) = 2\sum_i^ni = (n-1)^2$

\section{Exercises 5.4 - 6}

\begin{align*}
C_{1,1} &= M_1 + M_4 - M_5 + M_7\\
a_{00}b_{00}+ a_{01}b_{10} &= (a_{00} +a_{11} )(b_{00} +b_{11} )+ a_{11} (b_{10}−b_{00} ) - b_{11}(a_{00} +a_{01} ) +(a_{01}−a_{11} )(b_{10} +b_{11} )\\
C_{1,2} &= M_3 + M_5\\
a_{00}b_{01}+ a_{01}b_{11} &= a_{00}( b_{01}− b_{11})+( a_{00}+ a_{01}) b_{11}\\
C_{2,1} &= M_2 + M_4\\
a_{10}b_{00}+ a_{11}b_{11} &= (a_{10} +a_{11} )b_{00} +a_{11} (b_{10}−b_{00} )\\
C_{2,2} &= M_1 - M_2 + M_3 + M_6\\
a_{10}b_{01}+ a_{11}b_{11} &= (a_{00}  + a_{11} )(b_{00}  + b_{11} ) + a_{00} (b_{01} − b_{11} ) - b_{00}(a_{10}  + a_{11} )  + (a_{10} − a_{00} )(b_{00}  + b_{01} )
\end{align*}

All above are equations.

\section{Exercises 5.4 - 7}

\[
A_{00} =
\left[
  \begin{array}{cc}
    1 & 0\\
    4 & 1\\
  \end{array}
\right],
A_{01} =
\left[
  \begin{array}{cc}
    2 & 1\\
    1 & 0\\
  \end{array}
\right],
A_{10} =
\left[
  \begin{array}{cc}
   0  & 1\\
   5  & 0\\
  \end{array}
\right],
A_{11} =
\left[
  \begin{array}{cc}
   3  & 0\\
   2  & 1\\
  \end{array}
\right],\\
\]
\[
B_{00} =
\left[
  \begin{array}{cc}
   0  &1 \\
   2  & 1\\
  \end{array}
\right],
B_{01} =
\left[
  \begin{array}{cc}
   0  & 1\\
   0  & 4\\
  \end{array}
\right],
B_{10} =
\left[
  \begin{array}{cc}
    2 & 0\\
    1 & 3\\
  \end{array}
\right],
B_{11} =
\left[
  \begin{array}{cc}
   1  & 1\\
   5  & 0\\
  \end{array}
\right]
\]

\[
M1 = (A_{00} + A_{11})(B_{00} + B_{11}) =
\left[
  \begin{array}{cc}
   4 & 8\\
   20 & 14\\
  \end{array}
\right]
\]
\[
M2 = (A_{10} + A_{11})B_{00} =
\left[
  \begin{array}{cc}
   2 & 4\\
   2 & 8\\
  \end{array}
\right]
\]
\[
M3 = A_{00}(B_{01} − B_{11}) =
\left[
  \begin{array}{cc}
   -1 & 0\\
   -9 & 4\\
  \end{array}
\right]
\]
\[
M4 = A_{11}(B_{10} − B_{00}) =
\left[
  \begin{array}{cc}
   6 & -3\\
   3 & 0\\
  \end{array}
\right]
\]
\[
M5 = (A_{00} + A_{01})B_{11} =
\left[
  \begin{array}{cc}
   8 & 3\\
   10 & 5\\
  \end{array}
\right]
\]
\[
M6 = (A_{10} − A_{00})(B_{00} + B_{01}) =
\left[
  \begin{array}{cc}
   2 & 3\\
   -2 & -3\\
  \end{array}
\right]
\]
\[
M7 = (A_{01} − A_{11})(B_{10} + B_{11}) =
\left[
  \begin{array}{cc}
   3 & 2\\
   -9 & -4\\
  \end{array}
\right]
\]

\[
C_{00} = M_1 + M_4 - M_5 + M_7 =
\left[
  \begin{array}{cc}
   5 & 4\\
   4 & 5\\
  \end{array}
\right]
\]
\[
C_{01} = M_3 + M_5 =
\left[
  \begin{array}{cc}
   7 & 3\\
   1 & 9\\
  \end{array}
\right]
\]
\[
C_{10} = M_2 + M_4 =
\left[
  \begin{array}{cc}
   8 & 1\\
   5& 8\\
  \end{array}
\right]
\]
\[
C_{11} = M_1 - M_2 + M_3 + M_6 =
\left[
  \begin{array}{cc}
   3 & 7\\
   7 & 7\\
  \end{array}
\right]
\]

\[
C =
\left[
\begin{array}{cccc}
5& 4 & 7 &3 \\
4& 5 &1 &9\\
8& 1 &3&7\\
5& 8 &7&7\\
\end{array}
\right]
\]
\end{document}
